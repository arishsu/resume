\documentclass[languages=english,francais]{resume_cover_letter}

% infos
\firstname{Gaël}
\familyname{Foppolo}  
\title{Étudiant ingénieur @ Polytech Marseille}         
\address{}{\faHome~Marseille -- France}   
\email{me@gaelfoppolo.com}                      
\homepage{www.gaelfoppolo.com}
\mobile{+33 6 28 91 02 66}
\social[github]{gaelfoppolo}
\social[linkedin]{gaelfoppolo} 
\social[twitter]{gaelugio} 
\extrainfo{Permis B}
\photo[95pt][0pt]{img/photo}	
%\quote{Quote}

\begin{document}

\maketitle

\vspace*{-13mm}

\section{\faBriefcase~Expériences}

\vspace*{-1mm}

\tldatecventry{2017}{Stage - Ingénieur iOS \faApple~(3 mois)}{Care Labs}{Montpellier}{\url{www.carelabs.co}}{L'application \textit{CareBook} est un carnet de santé personnel, permettant de consulter ses rendez-vous, stocker ses documents ou mettre à jour son dossier médical. Ma mission a été la création complète de l'application : conception, développement, tests, recettage et déploiement. Disponible début septembre 2017.
}

\vspace*{2mm}

\tldatecventry{2015}{Stage - Développeur iOS \faApple~(3 mois)}{KeepCore}{Montpellier}{\url{www.keepcore.com}}{L'application \textit{HandiCarParking} a pour objectif de localiser rapidement les places de stationnement réservées aux personnes handicapées dans le monde entier. Elle se base sur les données d'\textit{OpenStreetMap}.\\
	Ma mission a été la création complète de l'application, des spécifications au déploiement en passant par le développement.
\newline{\url{https://github.com/gaelfoppolo/handicarparking}}}

\vspace*{-2mm}

\section{\faGavel~Projets}

\vspace*{-1mm}

\tldatecventry{2017}{AnalytiCeM}{Polytech}{Marseille}{\url{github.com/gaelfoppolo/AnalytiCeM}}{
AnalytiCeM a pour objectif de récupérer et d'analyser des données cérébrales, provenant d'un casque Muse. Pour cela, on utilise la notion de « session », séance pendant laquelle on récupère des données contextuelles (comme la météo) que l’on enrichit avec les signaux récupérés par le casque. On peut ainsi inférer des comportements en fonction du contexte.\newline{\url{} \hfill \textit{Technologies : Swift \& Realm}}}

\tldatecventry{2017}{Locomotor}{Polytech}{Marseille}{\url{github.com/gaelfoppolo/locomotor}}{
Locomotor est un comparateur de moyens de transport réels et fictifs. Il se base sur des critères prédéfinis que l'utilisateur peut sélectionner. Il génère une liste de véhicules qui correspondent \og le mieux \fg à la demande de l'utilisateur.
\newline{\url{} \hfill \textit{Technologies : Java \& MongoDB}}}

\vspace*{-3mm}

\section{\faGraduationCap~Formations}

\vspace*{-1mm}

\tlcventry{2015}{2018}
{Ingénieur informatique}{Polytech}{Marseille}{}{}

\tlcventry{2013}{2015}
{DUT Informatique}{Institut Universitaire de Technologie}{Montpellier}{}{}

\vspace*{-3mm}

\section{\faNewspaperO~Publication}

\vspace*{-1mm}

\begin{otherlanguage}{english}

\tldatecventry{2015}{Contextual Sequential Pattern Mining in Games: Rock, Paper, Scissors, Lizard, Spock}{Dumartinet, J., Foppolo, G., Forthoffer, L., Marais, P., Croitoru, M., \& Rabatel, J.}{Research and Development in Intelligent Systems XXXII (pp. 375-380)}{Springer International Publishing}{}

\end{otherlanguage}

\vspace*{-3mm}

\section{\faLanguage~Langues}

\vspace*{-1mm}

\begin{minipage}{\linewidth}
  \begin{multicols}{2}
    \centering{\textbf{Français} -- \faStar~\faStar~\faStar~\faStar~\faStarO\\}
    \vspace*{1mm}
    \centering{\textit{Projet Voltaire -- 80\%}}\\
    \columnbreak
    \centering{\textbf{Anglais} -- \faStar~\faStar~\faStar~\faStar~\faStarHalfO\\}
    \vspace*{1mm}
    \centering{\textit{TOEIC -- 935}}\\
  \end{multicols}
\end{minipage}

\vspace*{-2mm}

\section{\faGears~Compétences}

\vspace*{-2mm}

\begin{minipage}{\linewidth}
  \begin{multicols}{3}
    \centering{\textbf{Environnements\\}}
    \vspace*{2mm}
        
        	%\vspace*{3mm}
        %\centering{macOS~~\textsc{\faStar~\faStar~\faStar~\faStarHalfO~\faStarO}\\}
        \centering{iOS~~\textsc{\faStar~\faStar~\faStar~\faStarHalfO~\faStarO}\\}
    	\centering{Xcode~\textsc{\faStar~\faStar~\faStarHalfO~\faStarO~\faStarO}\\}
    	\centering{MongoDB~\textsc{\faStar~\faStar~\faStarO~\faStarO~\faStarO}\\}
    	
    \columnbreak
    \centering{\textbf{Langages\\}}
    \vspace*{2mm}
   
   \centering{Swift \& Obj-C~~\textsc{\faStar~\faStar~\faStar~\faStarO~\faStarO}\\}    
    \centering{Java~~\textsc{\faStar~\faStar~\faStar~\faStarO~\faStarO}\\}    
    \centering{SQL \& PL/SQL~~\textsc{\faStar~\faStar~\faStarHalfO~\faStarO~\faStarO}\\}
    	
       \columnbreak
    \centering{\textbf{Outils\\}}
    \vspace*{2mm}
    
    	%\vspace*{3mm}
    	
    	\centering{Git/CocoaPods~~\textsc{\faStar~\faStar~\faStar~\faStarHalfO~\faStarO}\\}
    	\centering{Slack/Trello~\textsc{\faStar~\faStar~\faStar~\faStar~\faStarHalfO}\\}
    	\centering{UML~~\textsc{\faStar~\faStar~\faStar~\faStarO~\faStarO}\\}
    
  \end{multicols}
\end{minipage}

%\vspace*{-1mm}

\section{\faBeer~Autres}

\vspace*{-1mm}

\tikzset{
    cercle/.pic={
      \node [draw, thick, circle, minimum width=10pt] {\tikzpictext};
    },
  }
\hspace*{1mm}
\begin{minipage}{\linewidth}
  \begin{tikzpicture}
  	\pic [pic text={\LARGE \faCoffee}] {cercle};
    \node[draw=none] at (0,-0.9) {Thé};
    \pic [pic text={\LARGE \faTelevision}] at (20mm,0) {cercle};
    \node[draw=none] at (2,-0.9) {Séries};
    \pic [pic text={\LARGE \faSpotify}] at (40mm,0) {cercle};
    \node[draw=none] at (4,-0.9) {OutRun};
    \pic [pic text={\LARGE \faLeaf}] at (60mm,0) {cercle};
    \node[draw=none] at (6,-0.9) {Jardinage};
    \pic [pic text={\LARGE \faApple}] at (80mm,0) {cercle};
    \node[draw=none] at (8,-0.9) {Apple};
    \pic [pic text={\LARGE \faBicycle}] at (100mm,0) {cercle};
    \node[draw=none] at (10,-0.9) {Running};
    \pic [pic text={\LARGE \faFirefox}] at (120mm,0) {cercle};
    \node[draw=none] at (12,-0.9) {Firefox};
    \pic [pic text={\LARGE \faMoonO}] at (140mm,0) {cercle};
    \node[draw=none] at (14,-0.9) {Nuit};
    \pic [pic text={\LARGE \faGithub}] at (160mm,0) {cercle};
    \node[draw=none] at (16,-0.9) {Open source};
  \end{tikzpicture}
\end{minipage}

\end{document}