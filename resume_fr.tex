\documentclass[languages=english,francais]{resume_cover_letter}

% infos
\firstname{Gaël}
\familyname{Foppolo}  
\title{Étudiant ingénieur en informatique}         
\address{}{\faHome~Marseille -- France}   
\email{me@gaelfoppolo.com}                      
\homepage{www.gaelfoppolo.com}
\mobile{+33 6 28 91 02 66}
\social[github]{gaelfoppolo}
\social[linkedin]{gaelfoppolo} 
\social[twitter]{gaelugio} 
\extrainfo{Permis B}
\photo[95pt][0pt]{img/photo}	

\begin{document}

\maketitle

\vspace*{-9mm}

\section{\faBriefcase~Expériences}

\tldatecventry{2016}{Stage - Développeur iOS \faApple}{Lundi Matin}{Montpellier}{\url{www.lundimatin.fr}}{\textit{RoverCash} remplace la caisse enregistreuse classique, transformant tablette et smartphone en véritable outil de gestion. L'application offre, en plus des fonctionnalités traditionnelles : annuaire, catalogue, vente, paiement, statistiques, etc.\\
	Mon objectif principal durant ce stage a été l’internationalisation (i18n) et la localisation
(l10n) de l’application.
\newline{\url{https://itunes.apple.com/app/id1061525820}}}

\vspace*{2mm}

\tldatecventry{2015}{Stage - Développeur iOS \faApple}{KeepCore}{Montpellier}{\url{www.keepcore.com}}{L'application \textit{HandiCarParking} a pour objectif de localiser rapidement les places de stationnement réservées aux personnes handicapées dans le monde entier. Elle se base sur les données d'\textit{OpenStreetMap}. Disponible en plusieurs langues, gratuitement sur l'App Store.
\newline{\url{https://itunes.apple.com/fr/app/id986777305}}}

\vspace*{-2mm}

\section{\faGavel~Projets}

\tldatecventry{2016}{Apprentissage supervisé}{Polytech}{Marseille}{\url{github.com/gaelfoppolo/examples-learning}}{Ce projet a pour but de trouver les similarités dans un jeu de données formatées et de proposer une ou plusieurs descriptions caractéristiques qui apportent le plus d'informations. L'application produit des solutions de la forme : \textit{\og un carré rouge de taille 8 à 10 est sur un triangle bleu ou vert de taille 5, lui-même à côté d'un cercle rouge de taille 2 à 4\fg}.}

\vspace*{-3mm}

\section{\faGraduationCap~Formations}

\tlcventry{2015}{0}
{Ingénieur informatique}{Polytech}{Marseille}{}{}

\tlcventry{2013}{2015}
{DUT Informatique}{Institut Universitaire de Technologie}{Montpellier}{}{}

\vspace*{-3mm}

\section{\faNewspaperO~Publication}

\begin{otherlanguage}{english}

\tldatecventry{2015}{Contextual Sequential Pattern Mining in Games: Rock, Paper, Scissors, Lizard, Spock}{Dumartinet, J., Foppolo, G., Forthoffer, L., Marais, P., Croitoru, M., \& Rabatel, J.}{Research and Development in Intelligent Systems XXXII (pp. 375-380)}{Springer International Publishing}{}

\end{otherlanguage}

\vspace*{-3mm}

\section{\faLanguage~Langues}

\vspace*{-1mm}

\begin{minipage}{\linewidth}
  \begin{multicols}{2}
    \centering{\textbf{Français\\}}
    \vspace*{2mm}
    \centering{\faStar~\faStar~\faStar~\faStar~\faStarO\\}
    \centering{\textit{Projet Voltaire -- 80\%}}\\
    \columnbreak
    \centering{\textbf{Anglais\\}}
    \vspace*{2mm}
    \centering{\faStar~\faStar~\faStar~\faStar~\faStarHalfO\\}
    \centering{\textit{TOEIC -- 935}}\\
  \end{multicols}
\end{minipage}

\vspace*{-2mm}

\section{\faGears~Compétences}


\begin{minipage}{\linewidth}
  \begin{multicols}{3}
    \centering{\textbf{Environnements\\}}
    \vspace*{2mm}
        \centering{macOS~~\textsc{\faStar~\faStar~\faStar~\faStarHalfO~\faStarO}\\}
        \centering{iOS~~\textsc{\faStar~\faStar~\faStar~\faStarHalfO~\faStarO}\\}
    \centering{Windows~~\textsc{\faStar~\faStar~\faStarO~\faStarO~\faStarO}\\}
    \centering{UNIX~~\textsc{\faStar~\faStar~\faStar~\faStarO~\faStarO}\\}
    \columnbreak
    \centering{\textbf{Langages\\}}
    \vspace*{2mm}
    \centering{Swift/Objective-C~~\textsc{\faStar~\faStar~\faStar~\faStarO~\faStarO}\\}
    \centering{Java~~\textsc{\faStar~\faStar~\faStarHalfO~\faStarO~\faStarO}\\}
    \centering{C/C++~~\textsc{\faStar~\faStar~\faStarHalfO~\faStarO~\faStarO}\\}
    \centering{SQL/JavaScript~\textsc{\faStar~\faStar~\faStar~\faStarO~\faStarO}\\}
    \columnbreak
    \centering{\textbf{Outils\\}}
    \vspace*{2mm}
    \centering{Xcode~\textsc{\faStar~\faStar~\faStar~\faStarHalfO~\faStarO}\\}
    \centering{Git/CocoaPods~~\textsc{\faStar~\faStar~\faStar~\faStarHalfO~\faStarO}\\}
    \centering{Slack/Trello~\textsc{\faStar~\faStar~\faStar~\faStar~\faStarHalfO}\\}
    \centering{UML~~\textsc{\faStar~\faStar~\faStar~\faStarO~\faStarO}\\}
  \end{multicols}
\end{minipage}

\section{\faBeer~Autres}

\vspace*{1mm}

\tikzset{
    cercle/.pic={
      \node [draw, thick, circle, minimum width=10pt] {\tikzpictext};
    },
  }
\hspace*{1mm}
\begin{minipage}{\linewidth}
  \begin{tikzpicture}
  	\pic [pic text={\huge \faCoffee}] {cercle};
    \node[draw=none] at (0,-1.1) {Café};
    \pic [pic text={\huge \faTelevision}] at (20mm,0) {cercle};
    \node[draw=none] at (2,-1.1) {Séries};
    \pic [pic text={\huge \faHeadphones}] at (40mm,0) {cercle};
    \node[draw=none] at (4,-1.1) {Musique};
    \pic [pic text={\huge \faLeaf}] at (60mm,0) {cercle};
    \node[draw=none] at (6,-1.1) {Jardinage};
    \pic [pic text={\huge \faApple}] at (80mm,0) {cercle};
    \node[draw=none] at (8,-1.1) {Apple};
    \pic [pic text={\huge \faGamepad}] at (100mm,0) {cercle};
    \node[draw=none] at (10,-1.1) {Jeux vidéo};
    \pic [pic text={\huge \faFirefox}] at (120mm,0) {cercle};
    \node[draw=none] at (12,-1.1) {Firefox};
    \pic [pic text={\huge \faMoonO}] at (140mm,0) {cercle};
    \node[draw=none] at (14,-1.1) {Nuit};
    \pic [pic text={\huge \faGithub}] at (160mm,0) {cercle};
    \node[draw=none] at (16,-1.1) {Open source};
  \end{tikzpicture}
\end{minipage}

\end{document}