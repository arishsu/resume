\documentclass[languages=english]{resume_cover_letter}

% infos
\firstname{Gaël}
\familyname{Foppolo}  
\title{Engineering student @ Polytech Marseille}         
\address{}{\faHome~Marseille -- France}   
\email{me@gaelfoppolo.com}                      
\homepage{www.gaelfoppolo.com}
\mobile{+33 6 28 91 02 66}
\social[github]{gaelfoppolo}
\social[linkedin]{gaelfoppolo} 
\social[twitter]{gaelugio} 
\extrainfo{Driving license}
\photo[95pt][0pt]{img/photo}	

\begin{document}

\maketitle

\vspace*{-13mm}

\section{\faBriefcase~Experiences}

\vspace*{-1mm}

\tldatecventry{2016}{Internship - iOS developer \faApple}{Lundi Matin}{Montpellier, France}{\url{www.lundimatin.fr}}{The \textit{RoverCash} application replaces the traditional cash register, turning your tablet and smartphone into a management tool. The app offers, besides the traditional functionnalities, a directory, a catalogue, payment handling, statistics gathering, etc.\\
	My main objective of my internship was the internationalization (i18n) and localization (l10n) of the app.
\newline{\url{https://itunes.apple.com/app/id1061525820}}}

\vspace*{2mm}

\tldatecventry{2015}{Internship - iOS developer \faApple}{KeepCore}{Montpellier, France}{\url{www.keepcore.com}}{The \textit{HandiCarParking} application aims to quickly locate parking spaces reserved to disabled people all over the world. It’s based on the \textit{OpenStreetMap} data. My assignment was the complete design of the app, from specifications to deployment.
\newline{\url{https://github.com/gaelfoppolo/handicarparking}}}

\section{\faGavel~Projects}

\vspace*{-1mm}

\tldatecventry{2017}{AnalytiCeM}{Polytech}{Marseille}{\url{github.com/gaelfoppolo/AnalytiCeM}}{
AnalytiCeM aims to retrieve and analyze brain data, coming from a Muse headband. In order to achieve that, we create "session", during which we retrieve contextual data (like weather) , which is augmented with brain data. We can then infer behaviors according to the context.\newline{\url{} \hfill \textit{Technology: Swift \& Realm}}}

\tldatecventry{2017}{Locomotor}{Polytech}{Marseille}{\url{github.com/gaelfoppolo/locomotor}}{
Locomotor is a real and fictional vehicles comparator. It is based on predefined criterias that the user can select. It outputs a list of the vehicles that fit "the most" the user's query. The comparator works in a client/server design, allowing multiple users to use the same dataset at the same time.
\newline{\url{} \hfill \textit{Technology: Java \& MongoDB}}}

\vspace*{-2mm}

\section{\faGraduationCap~Education}

\vspace*{-1mm}

\tlcventry{2015}{0}
{Engineering school in Computer Science}{Polytech}{Marseille}{France}{}

\tlcventry{2013}{2015}
{Associate's degree in Computer Science}{University Institute of Technology}{Montpellier}{France}{}

\vspace*{-3mm}

\section{\faNewspaperO~Publication}

\vspace*{-1mm}

\tldatecventry{2015}{Contextual Sequential Pattern Mining in Games: Rock, Paper, Scissors, Lizard, Spock}{Dumartinet, J., Foppolo, G., Forthoffer, L., Marais, P., Croitoru, M., \& Rabatel, J.}{Research and Development in Intelligent Systems XXXII (pp. 375-380)}{Springer International Publishing}{}

\section{\faLanguage~Languages}

\vspace*{-1mm}

\begin{minipage}{\linewidth}
  \begin{multicols}{2}
    \centering{\textbf{French} -- \faStar~\faStar~\faStar~\faStar~\faStarO\\}
    \vspace*{1mm}
    \centering{\textit{Projet Voltaire -- 80\%}}\\
    \columnbreak
    \centering{\textbf{Anglais} -- \faStar~\faStar~\faStar~\faStar~\faStarHalfO\\}
    \vspace*{1mm}
    \centering{\textit{TOEIC -- 935}}\\
  \end{multicols}
\end{minipage}

\section{\faGears~Skills}

\vspace*{-2mm}

\begin{minipage}{\linewidth}
  \begin{multicols}{3}
    \centering{\textbf{Environments\\}}
    \vspace*{2mm}
        
       %\vspace*{3mm}
        %\centering{macOS~~\textsc{\faStar~\faStar~\faStar~\faStarHalfO~\faStarO}\\}
        \centering{iOS~~\textsc{\faStar~\faStar~\faStar~\faStarHalfO~\faStarO}\\}
    	\centering{Xcode~\textsc{\faStar~\faStar~\faStarHalfO~\faStarO~\faStarO}\\}
    	\centering{MongoDB~\textsc{\faStar~\faStar~\faStarO~\faStarO~\faStarO}\\}
    	
    \columnbreak
    \centering{\textbf{Langages\\}}
    \vspace*{2mm}
    
    \centering{Swift \& Obj-C~~\textsc{\faStar~\faStar~\faStar~\faStarO~\faStarO}\\}    
    \centering{Java~~\textsc{\faStar~\faStar~\faStar~\faStarO~\faStarO}\\}    
    \centering{SQL \& PL/SQL~~\textsc{\faStar~\faStar~\faStarHalfO~\faStarO~\faStarO}\\}
    	
       \columnbreak
    \centering{\textbf{Outils\\}}
    \vspace*{2mm}
    
    	%\vspace*{3mm}
    	
    	\centering{Git/CocoaPods~~\textsc{\faStar~\faStar~\faStar~\faStarHalfO~\faStarO}\\}
    	\centering{Slack/Trello~\textsc{\faStar~\faStar~\faStar~\faStar~\faStarHalfO}\\}
    	\centering{UML~~\textsc{\faStar~\faStar~\faStar~\faStarO~\faStarO}\\}
    	
  \end{multicols}
\end{minipage}

\vspace*{1mm}

\section{\faBeer~Miscellaneous}

\vspace*{-1mm}

\tikzset{
    cercle/.pic={
      \node [draw, thick, circle, minimum width=10pt] {\tikzpictext};
    },
  }
\hspace*{1mm}
\begin{minipage}{\linewidth}
  \begin{tikzpicture}
%  	\hspace*{20mm}
  	\pic [pic text={\LARGE \faCoffee}] {cercle};
    \node[draw=none] at (0,-0.9) {Tea};
    \pic [pic text={\LARGE \faTelevision}] at (20mm,0) {cercle};
    \node[draw=none] at (2,-0.9) {TV series};
    \pic [pic text={\LARGE \faSpotify}] at (40mm,0) {cercle};
    \node[draw=none] at (4,-0.9) {OutRun}; 
    \pic [pic text={\LARGE \faLeaf}] at (60mm,0) {cercle};
    \node[draw=none] at (6,-0.9) {Gardening};   
    \pic [pic text={\LARGE \faApple}] at (80mm,0) {cercle};
    \node[draw=none] at (8,-0.9) {Apple};
    \pic [pic text={\LARGE \faBicycle}] at (100mm,0) {cercle};
    \node[draw=none] at (10,-0.9) {Running};
    \pic [pic text={\LARGE \faFirefox}] at (120mm,0) {cercle};
    \node[draw=none] at (12,-0.9) {Firefox};
    \pic [pic text={\LARGE \faMoonO}] at (140mm,0) {cercle};
    \node[draw=none] at (14,-0.9) {Night};
    \pic [pic text={\LARGE \faGithub}] at (160mm,0) {cercle};
    \node[draw=none] at (16,-0.9) {Open source};
  \end{tikzpicture}
\end{minipage}

\end{document}